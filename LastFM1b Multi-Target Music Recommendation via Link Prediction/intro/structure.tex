\section{Structure of Paper}

Throughout this body of work, I will address the objectives of the thesis in a sequential structure offering insight and understanding into the world of graph based learning for music recommendation. For this reason, this body of work will be outlined in the following structure

\subsection{State of the Art}

The state of the art will provide readers with relevant and necessary information to understand the research presented throughout the thesis. Emphasizing music recommendation, recent recommendation system applications and findings will be introduced. Afterwards a discussion outlining the available music recommendation data sets, limitations of the field, current findings, and research. As the state of the art continues, the introduction to graph based deep learning will allow this thesis to introduce the fundamental theory to achieve the fore mentioned objectives. Particularly outlined in this section, readers will be able to interpret not only how graph algorithms work, but how they are written in a generalized structure. This syntax is necessary for the reader to understand deep learning algorithms in the context of music recommendation systems, which will be utilized throughout the thesis.

\subsection{Methodology}
After the state of the art the background information and fundamental concepts to interpret the project should be clear. With the understanding of the objectives graph-based learning fundamentals accounted for. The methodology section will begin by framing the objectives into problems. The data set utilized for the thesis will be discussed much more in detail, including the advantages, limitations, and distributional information of the demographics of information. With the outlining of the information being used to create results, the methodology will begin its technical approach outline by describing the techniques utilized to complete the objectives. These objectives will be elaborated on as the implementation begins to discuss the technical specifics of each topic, and by the end of the methodology the reader will understand the formal actions taken to provide tested results.

\subsection{Results}
As the necessary requirements to understand the results section of the thesis should be made abundantly clear. The data, models, and evaluations should have been identified, and each reference justified. The results section will able the discussed topics and models presented through the thesis as they pertain to the analysis of link prediction within the LastFM1b data set. Additionally, the results of the LastFM1b data set will be utilized. The best preforming models will be selected to compute recommendations for all users within the heterogeneous graph. Once the recommendations have been acquired, this analysis should be able to provide a more comprehensive review of the performance capabilities of using graph based deep learning algorithms to recommend music content to users.
