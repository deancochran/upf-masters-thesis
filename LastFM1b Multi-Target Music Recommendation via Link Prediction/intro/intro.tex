\normallinespacing
\chapter{Introduction}

As technology continues to advance, there has been a dramatic explosion of digital content that is accessible online. Due to the sheer volume of accessible information, it is often overwhelming for users to find the right information. To alleviate the tedious act of filtering through online digital multimedia content, mathematical algorithms have been created to match relevant content, then score and rank the relevancy of the content with respect to the searching entity, creating recommendation systems. 

Recommendation systems filter through the most relevant pieces of content which provide structured recommendations which are relevant to the information it observes. \cite{Jannach2022} Though they are task specific and vary based off the setting they are used in, the specialized algorithms, in personalizing content for every user. Recommendation systems are such a powerful algorithm, that they must be monitored and structured properly. That's why for cases like the YouTube\cite{Davidson2010TheYV}, Netflix\cite{gomez2015netflix}, Pinterest \cite{ying2018graph} algorithms, where the recommendation is directly related to the service of the company, are used like products for the users to discover personalized content with.

These companies use this business model to advertise and tailor suggestions for their products to increase user retention and engagement, hence increasing revenue. \cite{lu2012recommender} As a result of this model, recommendation systems have been seen as potentially dangerous as they have the ability to influence users poorly. This ideology that recommendation systems have the capability to mislead users has led the academic community to research different ways for measuring bias, fairness, and privacy in recommendation systems. \cite{milano2020recommender}

Streaming platforms such as Spotify, Apple Music, or Amazon Music, are huge proponents of using recommendation systems. Music recommendation research is a competitive research field whose goal is to produce not only fair and unbiased music recommendations, but to provide users with relevant and interesting content. The main objective for these services is to retain loyalty of their users. \cite{jacobson2016music} The strategic mindset of competing companies like these, is to offer users a music recommendation service that is more engaging than the recommendation services provided by the other companies. 

The research field of music recommendation systems is largely supported by the academic community of music information retrieval (MIR), as well as the larger recommendation systems community. These communities have similar interests, which have often produced useful insights that modern day music streaming services utilize to form better music recommendation.

To provide examples of recent MIR research, newer discovering been able identified 3 styles of listening behaviors that consumers utilize MRS for; Basic Music Recommendation, Lean-in experiences, and Lean-back experiences. Where basic recommendation helps users find the content they may be searching for, Lean-in recommendation assists the engaged user typically associated with exploratory listening habits, and Lean-back recommendation helps users who wish to listen to music all day \cite{schedl2022music}. 

Music streaming services utilize a barrage of different methods to collect explicit and implicit data from user interactions. Explicit feedback includes information on when users like, rate, or save a piece of content, its very difficult to obtain a large summation of explicitly labeled content due to data sparsity issues. Since explicit feedback often is tedious to collect, implicit user feedback, or data collected on user listening events, clicks, skips are collected. As the user interacts with the platform more frequently, the preferences elicited from the listening behaviors of the user can be predicted. 

Often times this information is stored in an interaction database where a user will be labelled has having a connection to this piece of connect. This information does not indicate an explicit opinion from the user, rather as the user interacts with the platform more frequently, the preferences elicited from the listening behaviors of the user can be inferred. 

\section{Motivation}
The way music streaming services utilize explicit and implicit information also comes in a variety of forms. The most common approaches for music-based recommendations would be content and collaborative filtering methods. These have been utilized for decades and proven to be effective for finding content that is relevant for a given task. Specifically, to highlight collaborative filtering, it is a recommendation system approach which leverages the explicit information to predict interactions that users have not yet experienced but are calculated as likely to occur.

As recommendation systems have improved significantly in the last decade, users are often given personalized recommendations based on much more than just their explicit feedback with content. Modern day recommendation systems often use newer methodologies, complex mathematics, machine learning and/or artificial intelligence to alleviate the pitfalls of traditional content and collaborative filtering methods such as interaction user modeling, the cold-start problem, robustness, and the ability to explain why recommendations are made. \cite{Zhang2022}

The most notably for this thesis, modern recommendation system approach involves using graph-based deep learning. This unique direction includes methods such as nonlinear transformation, representation learning, and sequence modeling which have become widely successful as well as accepted by the academic community. \cite{Zhang2022} Though the method is quite resourceful, there are some apart deterrents of industrial level implementation due to the “black-box” phenomenon of deep learning, as well as the lack of mathematical transparency in the complex computations. \cite{Zhang2022} Uniquely however, graph based deep learning holds the natural capability to represent a user interactions (often represented as a bipartite graph). Continuous studies on graph-based learning have allowed for machine learning methods to come to light and are widely used for their modeling capabilities. \cite{Gao22} This method has not only changed the face of music recommendations, but it has also been used in other fields of research; from ETA prediction \cite{derrow2021eta} to molecular structure prediction \cite{li2009automated}. There have been numerous graph based recommendation methods have demonstrated there applicability in the music industry as well by utilizing common MRS data sets to preform a variety baseline evaluation measurements.\cite{li2020quaternion}\cite{fan2020relation}\cite{he2020mining}

A multimedia specialist and professor at Johannes Kepler University, Markus Shedl, offers a modern perspective on music recommendation systems by mentioning, “neural network architectures are still surprisingly sparsely adopted for music recommendation systems, even though the number of respective publications is increasing." \cite{Schedl2022} Furthermore, there has discussion of the apparent lack of established multi-modal data sets similar to the Million Song data set (MSD) \cite{bertin2011million} or the LFM-1b \cite{Schedl2016}. The lack of variety from within data sets, paired the large variety evaluation metrics within music recommendation research, makes the ability to reproduce published experiments or compare models proves to be quite challenging.

This high-level overview and structural breakdown of some modern music recommendation systems identifies the fundamentals needed to understand how new approaches, like graph neural networks in music recommendation are being addressed. With the paring of graph neural networks music recommendation systems have seen a lot of performance increase, yet many models are not evaluating there models on adequately large data sets to validate and exercises new graph based recommendation algorithms. In the following section, this thesis will address the simple objectives of the paper pertaining to this topic.

\subsection{DGL LFM1b}
The academic community of MIR has many available research data sets, however many of the collections are not suitable for music recommendation research. This leaves MIR researchers with a small collection of data sets for music recommendation research. Two of the most common music recommendation data sets, MSD \cite{bertin2011million} and Last FM \cite{dieleman2011audio}. These collections are often utilized in music recommendation research as they are easily accessible, and many industries outside of MIR  use the data for model evaluation. 

However, due to their lack of robustness, many researchers are not willing to utilize these data sets, as they may require more information besides what is offered in the original collection. In particular, for the Last FM data set, there exists a larger parent collection of sparse listening event histories for users. This data set is known as the LastFM1b data set \cite{Schedl2016}.

The LastFM1b is a collection of more than one billion listening events, intended to be used for various music retrieval and recommendation tasks. Specifically, LastFM1b provides more types of information on users and their listening behaviors that many other data sets do not provide. This additional information provides listening behaviors marked by timestamp for user interactions with additional types of entities (albums, tracks). 

With the understanding that music recommendation research on graph-based networks is not utilizing the LastFM1b dataset to evaluate their model's ability to generalize over heterogeneous graphs, but rather is using the smaller LastFM collection. It is worth exploring the capabilities and limits of utilizing more information than just users, artists, and tags in the case of the LastFM data set. Additionally, the popularity of deep learning with graphs to surely increase in the following years. This thesis will aim to create an optimized custom data set loaded for the LastFM1b, allowing researchers to easily acquire the necessary heterogeneous network that is required for graph-based deep learning research on the LastFM1b data set.

\subsection{Graph Based Deep Learning for Link Prediction}
As the development of deep learning in recommendation has seen a massive increase in novel graph based algorithms,  there has not been ample support by the MIR community to challenge, discuss or build their own implementations of novel deep learning algorithms for music recommendation. \cite{Schedl2022} \cite{Gao22} As mentioned, there is not much support for MRS deep learning research, so utilizing deep learning to contribute to the slowing research in music recommendation is required to maintain the topic's relevancy in the coming years. 

Among the novel recommendation algorithms, specifically within the field of deep learning, graph based learning methods have been shown to perform competitively against mode traditional recommendation methods.\cite{Gao2021} Therefore, with the knowledge of deep learning in MRS slowing \cite{Schedl2022}, and the successful graph based learning methods have shown in previous MRS collections, this thesis will aim to deploy Graph Neural Network models to perform heterogeneous link prediction on the LastFM1b data set. The models that will be attempted will be link prediction of user to track, user to album, and user to artist edges. 

\subsection{Beyond Accuracy Evaluation}

As link prediction is a common task for graph representation learning problems. As is the case link prediction is comparatively similar to recommendation systems. This thesis will conclude the contributions by applying the designed framework, and data set,  by implementing a  recommendation algorithm utilizing the resulting link prediction models and performance measures. From these recommendation algorithms, the collected information will allow for conclusions on the generated models utilizing the novel data set loader. 

These conclusions will offer further insight into not just the LastFM1b data set but user behavior that is learned to create the mentioned link prediction models. Additionally, there will be a discussion on evaluations of the proposed recommendation system. This paired with some non-accuracy statistics should proved additional insights into such measurements of diversity, novelty, and coverage to maintain interesting and relevant recommendation systems

\section{Structure of Paper}

Throughout this body of work, I will address the objectives of the thesis in a sequential structure offering insight and understanding into the world of graph based learning for music recommendation. For this reason, this body of work will be outlined in the following structure

\subsection{State of the Art}

The state of the art will provide readers with relevant and necessary information to understand the research presented throughout the thesis. Emphasizing music recommendation, recent recommendation system applications and findings will be introduced. Afterwards a discussion outlining the available music recommendation data sets, limitations of the field, current findings, and research. As the state of the art continues, the introduction to graph based deep learning will allow this thesis to introduce the fundamental theory to achieve the fore mentioned objectives. Particularly outlined in this section, readers will be able to interpret not only how graph algorithms work, but how they are written in a generalized structure. This syntax is necessary for the reader to understand deep learning algorithms in the context of music recommendation systems, which will be utilized throughout the thesis.

\subsection{Methodology}
After the state of the art the background information and fundamental concepts to interpret the project should be clear. With the understanding of the objectives graph-based learning fundamentals accounted for. The methodology section will begin by framing the objectives into problems. The data set utilized for the thesis will be discussed much more in detail, including the advantages, limitations, and distributional information of the demographics of information. With the outlining of the information being used to create results, the methodology will begin its technical approach outline by describing the techniques utilized to complete the objectives. These objectives will be elaborated on as the implementation begins to discuss the technical specifics of each topic, and by the end of the methodology the reader will understand the formal actions taken to provide tested results.

\subsection{Results}
As the necessary requirements to understand the results section of the thesis should be made abundantly clear. The data, models, and evaluations should have been identified, and each reference justified. The results section will able the discussed topics and models presented through the thesis as they pertain to the analysis of link prediction within the LastFM1b data set. Additionally, the results of the LastFM1b data set will be utilized. The best preforming models will be selected to compute recommendations for all users within the heterogeneous graph. Once the recommendations have been acquired, this analysis should be able to provide a more comprehensive review of the performance capabilities of using graph based deep learning algorithms to recommend music content to users.
