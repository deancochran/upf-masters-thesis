\section{LastFM1b Link Prediction}

% (image of recommendation of tracks, artists, albums)

As this thesis is focused on recommendation, it is also important to outline the fundamental correlation link prediction has with recommendation. Therefore, this section of the methodology must explain the task of link prediction the scope of HGNN, such the process can be used on the LastFM1b data set. Additionally, this section must explain how the compute predictions from the link prediction models can be utilized for music recommendation.

Recalling unique differences of the RGCN model when compared to its non relational GCN model, we can perform link prediction tasks on edges of multiple types. However, the presence of multiple relation types in a particular downstream task like link predicting proves to be a bit more cumbersome. Firstly, it should be observed that in many real-world scenarios modeled with heterogeneous graphs, the edge types have unbalanced distributions. Therefore, randomly splitting the edges into training sets may in fact remove an edge type from a given split of the graph. 

Additionally, due to the uniqueness, and distinct difference of a link prediction training process, when compared to a standard node classification process. It is beneficial to understand the common approaches utilized for training a model when preforming link prediction on heterogeneous graph.


\subsection{Training Process for Link Prediction}

Specifically, for graph neural network link prediction, there must be four splits made for training a deep learning model. Therefore, the requirement of a heterogeneous graph is to split all the edges of different types such that we have a balanced distribution of each edge type in a training, training supervision, validation, and test sets. 
\cite{CS224W}

% (image of link prediction)

For every target training supervision edge in the provided heterogeneous graph, we need to use the training edges to predict the likelihood of the given target edge. To compute the predictions with the training edges, we also must identify negative edges to perturb the supervision edge. \cite{CS224W} (the negative edges should not be belong to the training or supervision edges). With the training edges, a score can be assigned to the training supervision edge, and the negative edges. Specifically, each score represents the likelihood that the given negative or training supervision edge exists. 

Upon completion we can formulate a loss function which can be optimized to maximize the training supervision edge score and minimize the negative edge score.

% An example of a common loss function for heterogeneous link prediction would be:

% l = -log(o(fr3(he,ha)) - log(1-o(fr3(he,hb)))), where o is the sigmoid nonlinear activation function
% \cite{CS224W}

With the necessary information to compute the training stage of the model, at validation time, the use of all the training, and training supervision edges are utilized to predict the validation edges. Since the goal is to evaluate how each model can predict the existence of an edge of a specified type, we must compute the validation edge and negative edge scores. \cite{CS224W}

For each validation edge you can calculate the score, as well as the score of all the negative edges not in the set of training or training supervision edges. Upon completion you can rank the scores of all the edges and calculate metrics to evaluate the performance on the validation set, just as accomplished with the previous training set. 

Finally, to continue the testing phase, the training, training supervision, and validation edges are used to predict test edges against the negative edges not in the set of training or training supervision edges.



\subsection{Link Prediction for LastFM1b}

With the established training method for heterogeneous graphs to be carried out, the next objective to be outlined is how the thesis can approach testing link prediction models on the LastFM1b data set. Therefore, for each model type (RGCN and RHGNN), and for each type of link desired to be predicted (user-to-album, user-to-track, user-to-artist) a training, validating, and testing process must occur. 

Recalling the models discussed prior. The RHGNN model was a novel algorithm that incorporated some additional spectral information that models like RGCN would not be able compute. By following the for-mention link prediction training workflow, the RHGNN model can be applied to 3 different edges for link prediction. Therefore, for the following results chapter, there will be results corresponding to RHGNN user to track, user to album, and user to artist link prediction models. Additionally, the RGCN model will be utilized to perform a comparative study on the LastFM1b data set. Finally, upon the successful implementation of the three different link predictor models. A grid search approach will be applied to determine the most optimal performance score.
