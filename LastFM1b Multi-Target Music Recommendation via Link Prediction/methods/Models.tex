\section{Models}

The LastFM1b data set contains information on user listen behavior as its collections interactions that users have with artists, tracks, albums, and even indirectly the genres of these entities. Representing this information in a homogeneous graph cannot be done without sacrificing contextual and semantic information loss. Therefore, implementing this information in the form of a heterogeneous graph is a requirement.

Additionally, this limitation prevents the objectives of the thesis to be computed using a traditional Graph Convolutional Neural Network (GCN). As mentioned before, the original GCN model can only operate on homogeneous graphs. Therefore, to apply a neural network to a heterogeneous graph representation of the LastFM1b data set. A Heterogeneous Graph Neural Network (HGNN) will need to be utilized.

\subsection{Heterogeneous Graph Neural Networks (HGNNs)}
As discussed in the state of the art briefly, graphs can handle more than one edge type, and one node type. These forms of heterogeneous graphs can represent a wide variety real world scenarios.

% (HETEROGENEOUS GRAPH DEPICTION)

% G=(V,E,R,T) , where v is ... e is... r is... t is ...

% nodes vi in V
% node type T(vi) in T
% edges vi, r, vj in E where vi is the source node and vj is the destination node
% relation type r in R

In heterogeneous graphs, edges are represented as triples with a source node, relation type and a destination node. To further this thought, when two nodes are connected, there exists two directional edges two to and from the nodes. From this conceptualization of different entity types inside a heterogeneous graph, unique node and edge types can host different properties. 

To relate this to the thesis, heterogeneous graphs are particularly useful for modeling a recommendation system of users and items (typically in a bipartite graph), as it can model the multiple types of edges between the different node types. Utilizing heterogeneous graphs, instead of homogeneous graphs have led to a subdivision of graph based deep learning research. Where this particular subdivision studies how the additional information provided in heterogeneous graphs can be used to improve the performance of graph neural networks. \cite{wang2019kgat} The research field of statistical relational learning (SLR) applies directly to this subdivision. \cite{koller2007introduction} Such that, when using a heterogeneous graph, the analytical applications of SLR allow for better interpretations and explanations of the relational connections that exists between the nodes.


HGNN recommendation systems, leverage SLR to understand user interaction behavior implicitly. For music recommendation, commonly used data sets, like the LastFM data set, is used to validate the new research algorithms because the data set offers users, artists, and genres as different node types. As well as user interactions as edges between the users the artists, and genre relations to represent edges between artists and genres.\cite{aoscar2010music} However, the interpretations from additional relational connections, like connections users have with single tracks and albums, in context with the connections users have with artists, or artists have with genres, has yet to be explored. 

For this reason, this thesis will be utilizing two different HGNN algorithms to preform link prediction on the LastFM1b data set. In doing so the algorithms will be able to not only validate model performance but validate the importance of incorporating additional listening event information like tracks and albums, for music recommendation data sets as the LastFM1b has done. The two models that will be utilized in the link prediction tasks on the LastFM1b data set are described below in two parts.


\subsection{Relevant HGNN Methods and Algorithms}
HGNN recommendation systems, leverage statistical relational learning inside the deep learning algorithm. This is particularly why GCNs were introduced in the state of the art. Graph based models that can utilize heterogeneity to provide more expressive representations for nodes have been popularized through many different research publications. 

% (kipf, hamilton, jur L, others...) 

Specifically in this thesis, there are two algorithms that will be utilized to preform link prediction on the LastFM1b data set. Since these algorithms are different each other, as well as Graph Convolutional Networks, this thesis will discuss the two algorithms being used, as well as their differences.


\subsubsection{Relational Graph Convolutional Network (RGCN)}

% (PICTURE)

Relational Graph Convolutional Networks is an extension of the GCN model that can traverse over a graph with multiple edge types. Notably in the GCN paper it can be observed that the weight matrix, while being trainable, is shared amongst all the nodes in the graph. 

Specifically, for RGCN, there are unique trainable weight matrices for each edge type. This provides an importance factor to the nodes message during the propagation phase of the RGCN layer specified by the relation type. \cite{rgcn2017}


% The RGCN node wise update function:

% hi(l+1) = o(Wo(l)*hi(l) + sumfor-r-in-R(sumfor-j-in-Ni(1/cir * Wr(l)*hj(l))))

% In this equation Wo while not applying importance to the relational messages being aggregated for a updated node representation, it in fact give special treatment importance factoring to the self type of the node, preserving the information that exists in the input node

RGCNs will have more parameters than GCNs, this is due to the increase of relation weight matrices. This inevitably provides us with the need to regularize the rapidly increase parameters with each new edge type that exists in the given graph. The RGCN paper provides two solutions to regularize the weights.

The first method being basis decomposition, where the number unique weights are specified, then the unique weights are then scaled to the specified number to bound the amount of weight matrices.


% Wr(l)=sum-for-b-in-B(arb(l)*Vb(l))

Secondly the next method proposed is block diagonal decomposition, this takes the relational weight matrices and stacks each in a larger empty matrix diagonally. The intuition behind this follows the concept that some nodes messages might hold features that are specific for a particular clustering or grouping. With block diagonal decomposition this new singular shared weight matrix can address those nodes in different clusters appropriately.

Notice that within this example it was not expressed that is algorithm had the capability to traverse a graph of multiple node types as well as multiple edge types. In a heterogeneous approach where a graph multiple node types as well as multiple edge types, RGCN can still be applied by using a heterogeneous operation wrapper. This is simply an operation that allows models that can handle multiple edge types, to be able to handle multiple node types as well. The operation applies multiple instances of the model, in this case RGCN, to aggregate incoming information from nodes of different types.

As a final addition, to the information concerning the RGCN model. The OpenHGNN framework mentioned in the state of the art provides implementation of the algorithm with a heterogeneous wrapper. This implementation can be utilized later in the training and testing process of link prediction models, to validate other performance findings presented in this thesis

\subsubsection{Heterogeneous Graph Representation Learning with Relation Awareness (RHGNN)}
Notably of the constantly improving HGNN models, there are very few which aim to corporate the factor of edge representations into the downstream node embedding representations. This concept was challenged by the researchers studying Heterogeneous Graph Representation Learning with Relation Awareness \cite{rhgnn2021} in 2021. 

% (Picture of RHGNN)

The authors determined that it substantially important to not just learn the representations of edge relations, but also node representations with respect to different there relational interconnections. This builds upon the fundamental HGNN model by incorporating a RGCN with a heterogeneous wrapper component, utilizing a cross-relation operation to improve node representations, and proposing a fusing operation to aggregate relation-aware node representations into a single low dimensional embedding\cite{rhgnn2021}


From the authors findings, and their results of RHGNN performance amongst other models, they observed a noteworthy performance increase over the standard baseline models including RGCN.

Similarly, to the last note of the RGCN model, the RHGNN model is provided as an implementation within the OpenHGNN code base, as well as provided by the authors code using the deep graph library framework. Therefore, open source implementation can be utilized later in the training and testing process of link prediction models.


\newpage