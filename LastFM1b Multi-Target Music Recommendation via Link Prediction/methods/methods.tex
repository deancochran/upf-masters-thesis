\chapter{Methods}
Form the understandings of modern graph based deep learning music recommendation systems, as well the known limitations that are associated with these approaches as they relate to music recommendation systems, this chapter will address the technical specifics to form methodology of the thesis.

Firstly, this methodology section will outline the known data that will be utilized for the thesis. The LastFM1b data set will be formally introduced with the distributional analysis conducted by the published paper, as well as some additional supplementary information on converting the data into a data loader for the specified framework.

Recalling the state of the art chapter, this methodology will summarize the models selected to be used for testing. As well as offer insights into heterogeneous graph neural networks and how they specific pertain to music recommendation.

As training for link prediction is a unique approach in graph based machine learning, the chapter will continue into discussing how these models will be trained. In doing, so the chapter will address how recommendation applies to this method of prediction, and how it can be utilized in the context of music recommendation for the LastFM1b data set. Specifically, the recommendation model will utilize three different link prediction models for scoring likely interactions among users and unseen tracks, albums, and artists.

Finally these approaches will need to be qualitatively and quantitatively justified, the chapter will finish by providing insights on the evaluation metrics utilized to interpret the performance of the LFM1b data set loader, the link prediction models, as well as the results from the recommendations provided from the link prediction models. 


\section{DGL LFM1b Data set}

In the field of MRS there are a few well know data sets that have risen to popularity. Most notably, the Million song data set (MSD) and LastFM data set have been utilized the most in MRS research. Particularly with the LastFM data set, there aren't enough usable versions applicable for a comparative graph based deep learning music recommendation approach. As a result, this limitation has formed a barrier in the ability to reproduce many modern recommendation system papers, especially for beginner researchers. 

\subsection{LastFM1b Data set}

The LastFM1b data set provides more information that can be utilized in a deep learning scenario. Therefore, to alleviate the issue, I will outline the methodological approaches used to create an optimized method for generating a LastFM1b music recommendation data loader in the deep graph library frame. 

Uniquely, the LastFM1b is not utilized as often as its famous predecessor LastFM data sets. This can be speculated on, however regardless of the applications of the different data sets, the variety of the recorded information is what makes the LastFM1b data set useful for this project. 

As the traditional LastFM data sets offer information on user to artists listening behaviors. The LastFM1b data set also incorporates user listening behavior for tracks and albums. The significance of incorporating this information in large deep learning graphs to discern an observable performance difference between the impact of incorporating albums and tracks is not well studied. \cite{hgRepAndApp2022} \cite{Schedl2016}

As a result, the LastFM data set has been utilized much more than the modernized LastFM1b data set in recent years. In fact, the LastFM1b data set is indeed outdated. The LastFM2b data set was released in September of 2021 and offers even more information than that of the 1b data set. This will be addressed later in the conclusions of this thesis.


Given that of the available data sets in the most popular graph neural network frameworks do not utilize the LastFM1b data set. Implementing a custom data loader, that can be utilized by graph based deep learning communities, increase the necessary exposure the MRS community needs to stay relevant in the field of graph based deep learning, but it would provide contextual information of user interaction behavior with more than users and artist connections. In combination this would allow the music recommendation research community to utilize and understand the impact of diversifying the relational information that is collected in the LastFM1b data set.

\subsection{Data Loading}

For the given objective, a custom in memory Deep Graph Library (DGL) data set would properly expose the music recommendation community to one of the most popular graph based deep learning communities. To implement this, a large devotion to deep graph library framework is a prerequisite. The reasoning for this choice is due to the already existing LFM4k data set in the Pytorch geometric library, therefore to diversify the exposure of data sets amongst professionals, selecting a framework that does not have a music recommendation data set loader is a proper choice.

Before outlining the implementation with respect to the DGL framework, its important to understand the demographics and structure of the LastFM1b data set which can be downloaded at the publishing website's location. Inside the downloadable zip folder there are text files corresponding to the following information: users, artists, albums, tracks, and the one billion listening events. Additionally, the data set can be downloaded with the accompaniment of the LastFM1b UGP data set which analyzes the tags or genres of the artists, and users. The authors of the data set have published a distributional analysis on the website which is useful for those interested in the specific features and structure of the physical data set. \cite{Schedl2016}

As discussed in the state of the art, the deep graph library (DGL) provides a popularize frame for computing graph-based machine learning tasks on complex heterogeneous graphs. In DGL a customizable class made for custom in memory data sets can be used properly when a handful of key functions are. These include a load(), save(), and a process() function. The save() and load() functions simply work with reading and writing the final representations of the compiled heterogeneous graph to and from memory. The process() function however is the brains of the data loading functions.

When the process() function is called a large number of operations are required to read through the LastFM1b data set and compile one singular graph data object. Specifically these processes can be divide into three parts. \subsection{Compiling DGL Graph Data} To load a heterogeneous graph using the python deep graph library framework, it is required to collect a dictionary of every edge that exists in the graph. 


% The structure of a graph data dictionary is as follows

% graph data ={

% 'node_type': tensor(source_node_ids), tensor(destination_node_ids), 
% ...
% ...

% }

% Where all the ids for every node start at 0 and end at Nt, and N is the number of nodes of node type t. 

Notice that this information is not structured like a typical adjacency matrix, this is due to the large storage space that it requires to compile an adjacency matrix of millions if not billions of nodes. Therefore, an adjacency list is not so harsh on memory allocation.

% hg = dgl.HeteroGraph(graphData)
% (Picture of output)

As a technical note on the structural representation of the graph. There is not an edge connection provided for track to albums. The justification for this resides in the structure of the LastFM1b files.  This is particularly not computed since the information of track and album occurrences can only be evaluated through filtering of the listen events file. This is uniquely cumbersome as it is not listed in the track or album listings.

\subsection{Compiling DGL Node Data}
Once the graph data has been loaded into memory, the next task before the processing function finishes, is to add features to the nodes and edges of the graph. This step can be done in a variety of different ways. Notably at this step in the process, since the LastFM1b provided just the names of the artists, tracks, and albums, we can't not utilize audio or other digital signal processing methodologies to provide input embedding representations for the artist, album, or track nodes. As the information that is provided in the LastFM1b is not particularly expressive for the artists, albums, or tracks. Utilizing the limited features for node features for input into our models would not provide the optimal outcome in our results. As a result, there are no expressive features that can be utilized to compute node embedding representations through machine learning. However as described earlier in the thesis, implicit information is often utilized in recommendation systems as explicit information is traditionally sparse in nature.


\subsubsection{Compiling DGL Node Data}
% (Picture of Metapath2vec)

The metapath2vec method \cite{dong2017metapath2vec} was published in 2017 to challenge this issue and propose a solution to finding a resourceful way to preserve node context in heterogeneous networks node embedding representations. The published work offered a solution to the issue of heterogeneous graphs having irregular, or sparse collections of explicit features. The preserved node context in the final embedding representations were observed to improve downstream graph learning tasks like node classification, link prediction, clustering, and graph classification. The metapath2vec method can capture semantic and structural relationships between different types of nodes by preforming random walks operations over a specified heterogeneous meta path. As a result, many different heterogeneous graph models utilize the final embedding representations of the nodes generated by the metapath2vec model as inputs into a deep learning model for different prediction tasks.  \cite{dong2017metapath2vec}


% hg.nodes[nodeType].data[featureName]=metapath2vec.embeddings[nodeType]

The metapath2vec algorithm can be applied to the LastFM1b graph object to compute high quality node embedding representations as inputs into a deep learning model. Allowing for our later discussed deep learning models to utilize the node features that are structurally and graphically aware to each other for downstream prediction tasks.

As a note on metapath2vec, users and items interactions should be more emphasized than other types of interactions in the graph. This is because for our recommendation task, the objective is to provide recommendation of artists, albums, and tracks to users. Therefore, only user item meta-paths are selected for the random walk sequence, which has been shown to improve downstream node embedding representations. \cite{hgRepAndApp2022}


\subsection{Compiling DGL Edge Data}
Different than the previously described node representations, the edge data must be added to the graph. Specifically, two different approaches that can be made to add edge data to our graph. Firstly, in a user listen database an edge from a user to an artist, album, or track can be represented by multiple edges, each denoted with a timestamp. Alternatively, the edges from a particular user can be denoted as one edge with a weight of the number of interactions had with (number of times the user listened to) a particular artist, album, or track. Additionally, this value can be normalized as it is true that some users will listen specific artists, albums, or tracks in an unbalanced manner.

Therefore, this implementation of a DGL LastFM1b Data loader offers a compiled graph compatible with either form of edge representation. To specify which form of representation is needed to compile, bash script arguments can be added to commands.


% (image of different edge data methods)

% hg.edges[edgeType].data[normPlaycount]=normPlaycounts
% hg.edges[edgeType].data[playcount]=playcounts

% or 

% hg.edges[edgeType].data[timestamp]=timestamps

As a notice on the play count and timestamp edge data implementation, the weight values and timestamps only exist for edges connection users to artists, albums, and tracks. For edges in the heterogeneous graph that are outside of this set, there is just a tensor of ones to represent no weight, or no information at all to represent no timestamp.
\section{Models}

The LastFM1b data set contains information on user listen behavior as its collections interactions that users have with artists, tracks, albums, and even indirectly the genres of these entities. Representing this information in a homogeneous graph cannot be done without sacrificing contextual and semantic information loss. Therefore, implementing this information in the form of a heterogeneous graph is a requirement.

Additionally, this limitation prevents the objectives of the thesis to be computed using a traditional Graph Convolutional Neural Network (GCN). As mentioned before, the original GCN model can only operate on homogeneous graphs. Therefore, to apply a neural network to a heterogeneous graph representation of the LastFM1b data set. A Heterogeneous Graph Neural Network (HGNN) will need to be utilized.

\subsection{Heterogeneous Graph Neural Networks (HGNNs)}
As discussed in the state of the art briefly, graphs can handle more than one edge type, and one node type. These forms of heterogeneous graphs can represent a wide variety real world scenarios.

% (HETEROGENEOUS GRAPH DEPICTION)

% G=(V,E,R,T) , where v is ... e is... r is... t is ...

% nodes vi in V
% node type T(vi) in T
% edges vi, r, vj in E where vi is the source node and vj is the destination node
% relation type r in R

In heterogeneous graphs, edges are represented as triples with a source node, relation type and a destination node. To further this thought, when two nodes are connected, there exists two directional edges two to and from the nodes. From this conceptualization of different entity types inside a heterogeneous graph, unique node and edge types can host different properties. 

To relate this to the thesis, heterogeneous graphs are particularly useful for modeling a recommendation system of users and items (typically in a bipartite graph), as it can model the multiple types of edges between the different node types. Utilizing heterogeneous graphs, instead of homogeneous graphs have led to a subdivision of graph based deep learning research. Where this particular subdivision studies how the additional information provided in heterogeneous graphs can be used to improve the performance of graph neural networks. \cite{wang2019kgat} The research field of statistical relational learning (SLR) applies directly to this subdivision. \cite{koller2007introduction} Such that, when using a heterogeneous graph, the analytical applications of SLR allow for better interpretations and explanations of the relational connections that exists between the nodes.


HGNN recommendation systems, leverage SLR to understand user interaction behavior implicitly. For music recommendation, commonly used data sets, like the LastFM data set, is used to validate the new research algorithms because the data set offers users, artists, and genres as different node types. As well as user interactions as edges between the users the artists, and genre relations to represent edges between artists and genres.\cite{aoscar2010music} However, the interpretations from additional relational connections, like connections users have with single tracks and albums, in context with the connections users have with artists, or artists have with genres, has yet to be explored. 

For this reason, this thesis will be utilizing two different HGNN algorithms to preform link prediction on the LastFM1b data set. In doing so the algorithms will be able to not only validate model performance but validate the importance of incorporating additional listening event information like tracks and albums, for music recommendation data sets as the LastFM1b has done. The two models that will be utilized in the link prediction tasks on the LastFM1b data set are described below in two parts.


\subsection{Relevant HGNN Methods and Algorithms}
HGNN recommendation systems, leverage statistical relational learning inside the deep learning algorithm. This is particularly why GCNs were introduced in the state of the art. Graph based models that can utilize heterogeneity to provide more expressive representations for nodes have been popularized through many different research publications. 

% (kipf, hamilton, jur L, others...) 

Specifically in this thesis, there are two algorithms that will be utilized to preform link prediction on the LastFM1b data set. Since these algorithms are different each other, as well as Graph Convolutional Networks, this thesis will discuss the two algorithms being used, as well as their differences.


\subsubsection{Relational Graph Convolutional Network (RGCN)}

% (PICTURE)

Relational Graph Convolutional Networks is an extension of the GCN model that can traverse over a graph with multiple edge types. Notably in the GCN paper it can be observed that the weight matrix, while being trainable, is shared amongst all the nodes in the graph. 

Specifically, for RGCN, there are unique trainable weight matrices for each edge type. This provides an importance factor to the nodes message during the propagation phase of the RGCN layer specified by the relation type. \cite{rgcn2017}


% The RGCN node wise update function:

% hi(l+1) = o(Wo(l)*hi(l) + sumfor-r-in-R(sumfor-j-in-Ni(1/cir * Wr(l)*hj(l))))

% In this equation Wo while not applying importance to the relational messages being aggregated for a updated node representation, it in fact give special treatment importance factoring to the self type of the node, preserving the information that exists in the input node

RGCNs will have more parameters than GCNs, this is due to the increase of relation weight matrices. This inevitably provides us with the need to regularize the rapidly increase parameters with each new edge type that exists in the given graph. The RGCN paper provides two solutions to regularize the weights.

The first method being basis decomposition, where the number unique weights are specified, then the unique weights are then scaled to the specified number to bound the amount of weight matrices.


% Wr(l)=sum-for-b-in-B(arb(l)*Vb(l))

Secondly the next method proposed is block diagonal decomposition, this takes the relational weight matrices and stacks each in a larger empty matrix diagonally. The intuition behind this follows the concept that some nodes messages might hold features that are specific for a particular clustering or grouping. With block diagonal decomposition this new singular shared weight matrix can address those nodes in different clusters appropriately.

Notice that within this example it was not expressed that is algorithm had the capability to traverse a graph of multiple node types as well as multiple edge types. In a heterogeneous approach where a graph multiple node types as well as multiple edge types, RGCN can still be applied by using a heterogeneous operation wrapper. This is simply an operation that allows models that can handle multiple edge types, to be able to handle multiple node types as well. The operation applies multiple instances of the model, in this case RGCN, to aggregate incoming information from nodes of different types.

As a final addition, to the information concerning the RGCN model. The OpenHGNN framework mentioned in the state of the art provides implementation of the algorithm with a heterogeneous wrapper. This implementation can be utilized later in the training and testing process of link prediction models, to validate other performance findings presented in this thesis

\subsubsection{Heterogeneous Graph Representation Learning with Relation Awareness (RHGNN)}
Notably of the constantly improving HGNN models, there are very few which aim to corporate the factor of edge representations into the downstream node embedding representations. This concept was challenged by the researchers studying Heterogeneous Graph Representation Learning with Relation Awareness \cite{rhgnn2021} in 2021. 

% (Picture of RHGNN)

The authors determined that it substantially important to not just learn the representations of edge relations, but also node representations with respect to different there relational interconnections. This builds upon the fundamental HGNN model by incorporating a RGCN with a heterogeneous wrapper component, utilizing a cross-relation operation to improve node representations, and proposing a fusing operation to aggregate relation-aware node representations into a single low dimensional embedding\cite{rhgnn2021}


From the authors findings, and their results of RHGNN performance amongst other models, they observed a noteworthy performance increase over the standard baseline models including RGCN.

Similarly, to the last note of the RGCN model, the RHGNN model is provided as an implementation within the OpenHGNN code base, as well as provided by the authors code using the deep graph library framework. Therefore, open source implementation can be utilized later in the training and testing process of link prediction models.


\newpage
\section{Training Graph Neural Networks}
Due to the intricacies of graph neural networks, there are some noteworthy training differences that need to be used specifically for large graph representations like the DGL LastFM1b heterogeneous graph. These differences in training will be briefly outlined in the following sections, with additional task specific information to be outlined later.

\subsection{Training for large graphs}
As it can be observed from many of the possible scenarios that heterogeneous graph networks can be applied to. Common limitations of these networks reside in the number of computations required to compute predictions. The LastFM1b graph for instance has millions of nodes and billions of edges. \cite{Schedl2016}

This can severely limit the ability to train a graph network model as a naive full batch machine learning approach must load the full graph in memory, compute new embedding representations for every node, for each GNN layer, compute the loss, and preform gradient descent to optimize the scoring function. This is highly inefficient to do, and cause limitations when attempting to utilize GPUs of limited storage capacity. As a solution mini-batch training is often utilized to efficiently increase the computation speed of large graph networks.

\subsection{Mini batch training}For the specified training spilt of the graph, a series of computations for each node needs to be meet. For each target node, the neighbors of the node are used in the message passing aggregation and update function. The objective is to fit the necessary computations for all the target nodes into a particular batch of the training data. However, this may not always be feasible to fit each target node's small graphs in a huge computation graph due to the high connectivity of a certain target node's neighbors. For this reason, GNNs employ a sampling function, such that a subset of the target node's neighbors is used. This allows the computation graph to be bounded in size such that they can be utilized for large network training.

\subsubsection{Neighbor Sampling}
As discussed above in, graph networks generate a single node embedding representations utilizing aggregation functions. Within the context of a neural network layer, the aggregation search depth utilized to update of a target node's embedding representations increases with each layer in the network. As an example, a GNN with two layers generates embed dings for its nodes using 2-hop neighborhood structure. As a generalization of this example, you can say that a k-layer GNN generates an embedding for its nodes using K-hop neighborhood structure.


Notice that for each target node that is updated, only K-hop neighbors are required to interpret the updated representation. This is a core understanding to interpret the reason for why GNNs use neighborhood sampling for faster computational speed and less memory allocation.
\section{LastFM1b Link Prediction}

% (image of recommendation of tracks, artists, albums)

As this thesis is focused on recommendation, it is also important to outline the fundamental correlation link prediction has with recommendation. Therefore, this section of the methodology must explain the task of link prediction the scope of HGNN, such the process can be used on the LastFM1b data set. Additionally, this section must explain how the compute predictions from the link prediction models can be utilized for music recommendation.

Recalling unique differences of the RGCN model when compared to its non relational GCN model, we can perform link prediction tasks on edges of multiple types. However, the presence of multiple relation types in a particular downstream task like link predicting proves to be a bit more cumbersome. Firstly, it should be observed that in many real-world scenarios modeled with heterogeneous graphs, the edge types have unbalanced distributions. Therefore, randomly splitting the edges into training sets may in fact remove an edge type from a given split of the graph. 

Additionally, due to the uniqueness, and distinct difference of a link prediction training process, when compared to a standard node classification process. It is beneficial to understand the common approaches utilized for training a model when preforming link prediction on heterogeneous graph.


\subsection{Training Process for Link Prediction}

Specifically, for graph neural network link prediction, there must be four splits made for training a deep learning model. Therefore, the requirement of a heterogeneous graph is to split all the edges of different types such that we have a balanced distribution of each edge type in a training, training supervision, validation, and test sets. 
\cite{CS224W}

% (image of link prediction)

For every target training supervision edge in the provided heterogeneous graph, we need to use the training edges to predict the likelihood of the given target edge. To compute the predictions with the training edges, we also must identify negative edges to perturb the supervision edge. \cite{CS224W} (the negative edges should not be belong to the training or supervision edges). With the training edges, a score can be assigned to the training supervision edge, and the negative edges. Specifically, each score represents the likelihood that the given negative or training supervision edge exists. 

Upon completion we can formulate a loss function which can be optimized to maximize the training supervision edge score and minimize the negative edge score.

% An example of a common loss function for heterogeneous link prediction would be:

% l = -log(o(fr3(he,ha)) - log(1-o(fr3(he,hb)))), where o is the sigmoid nonlinear activation function
% \cite{CS224W}

With the necessary information to compute the training stage of the model, at validation time, the use of all the training, and training supervision edges are utilized to predict the validation edges. Since the goal is to evaluate how each model can predict the existence of an edge of a specified type, we must compute the validation edge and negative edge scores. \cite{CS224W}

For each validation edge you can calculate the score, as well as the score of all the negative edges not in the set of training or training supervision edges. Upon completion you can rank the scores of all the edges and calculate metrics to evaluate the performance on the validation set, just as accomplished with the previous training set. 

Finally, to continue the testing phase, the training, training supervision, and validation edges are used to predict test edges against the negative edges not in the set of training or training supervision edges.



\subsection{Link Prediction for LastFM1b}

With the established training method for heterogeneous graphs to be carried out, the next objective to be outlined is how the thesis can approach testing link prediction models on the LastFM1b data set. Therefore, for each model type (RGCN and RHGNN), and for each type of link desired to be predicted (user-to-album, user-to-track, user-to-artist) a training, validating, and testing process must occur. 

Recalling the models discussed prior. The RHGNN model was a novel algorithm that incorporated some additional spectral information that models like RGCN would not be able compute. By following the for-mention link prediction training workflow, the RHGNN model can be applied to 3 different edges for link prediction. Therefore, for the following results chapter, there will be results corresponding to RHGNN user to track, user to album, and user to artist link prediction models. Additionally, the RGCN model will be utilized to perform a comparative study on the LastFM1b data set. Finally, upon the successful implementation of the three different link predictor models. A grid search approach will be applied to determine the most optimal performance score.



\section{LastFM1b Link Prediction as recommendation}

Recommendation systems are not simply a link prediction model, however there are some principal correlations that graph based recommendation systems share with graph-based link prediction tasks. As mentioned prior there are constantly evolving methods for graph based recommendation systems \cite{Gao2021}. Specifically, HGNN recommendations models are in a constant state of change. To this extent, the objective of the thesis is to make artist, album, or track recommendations for users. Therefore, a scoring functioning to evaluate the probability of all unseen edges from a source node user, to an artist, album, or track destination node is required. There are different scoring techniques to identify a more expressive likelihood score of an edge existing.\cite{Kumar20} However, given that the implementation of three deep learning link predicting models have been proposed to score the likelihood of unseen edges for user-to-album, user-to-track, and user-to-artist interaction, these probabilities can be utilized as a likelihood score for personalized recommendation.


\subsection{Link Prediction as recommendation}

As mentioned above the link prediction models that are to be trained to predict specifically, the existence of unseen user-to-album, user-to-track, and user-to-artist interactions will be used to make recommendations for all user nodes in each graph. In order create recommendations, the highest preforming link prediction model for each link type (user-to-album, user-to-track, and user-to-artist), will iterate over all the specific user and destination node pairs relative to the link type. For every unique pairing of user and destination node, the respective trained link prediction model will score the likely hood the edge's likelihood of existing. Once complete, a selection of the top-K scored edges of a specified edge type for any user can be presented as recommendations. This process can be completed for ever edge type with a user source node (user-to-album, user-to-track, and user-to-artist).

Additionally, depending on the scoring likelihood of the top-K recommendations for each edge type, a ranked combination list of the three recommendation lists at top-K, could perform a multi-targeted recommendation for any specified user in the LastFM1b data set.

% Evaluation (metrics, hyperparameters optimization, etc.)
\section{Evaluation}

To conclude the chapter on logistical approaches used for achieve the objectives of the thesis. As well to provide validation for the proposed methods chosen. Evaluation of the various results provided by the LastFM1b data loader, the LastFM1b link prediction models, and the LastFM1b recommendations is required.

In order to provide justification for the choose evaluation measures. This section of the methodology will identify the chosen quantitative and qualitative metrics that will be used to evaluate this thesis.

\subsection{Evaluation of the LastFM1b Data loader}

The LastFM1b data loader implemented in the Deep Graph Library frame provides many customizes features. Most notably, the data loader provides the option of specifying a subset of the types of nodes desired in the compiled graph. The data loader also provides additional subset capabilities as the number of users represented in the compiled graph can be specified. 

These complementary features are resourceful for researchers, as compiling the full LastFM1b heterogeneous graph in memory can take quite a long time to compute. To list the remaining features of the data loader: the metapath2vec classifier for the node embedding representations has customizable hyper parameters, the metapath2vec classifier can be turned off, the GPU to compile the information can be specified, and the overwrite functions can be ignored if specified.

To evaluate this data set and the capabilities it holds, a distributional comparison study of the DGL data loader can be performed. A comparison is particularly required, as to compile the full LastFM1b data set or any of the subsets made available. There are a significant number of removed instances of artists, tracks, albums, and listen events from the original data set. The reasoning for doing so, was so that the resulting LastFM1b data from the DGL data loader could represent a fully connected graph.


\subsection{Evaluation of the RHGNN Link Prediction}

The purpose for evaluating the link models trained on the LastFM1b data set, is not only to validate the model, rather it is also to validate the significance of incorporating additional contextual information inside a heterogeneous, ergo not just graph between users, artists, and genres.

With this objective driving the primary focus of evaluating link prediction on the LastFM1b data set there are a collection of measures that should be presented a deep learning models’ effectiveness to predict. To evaluate the link prediction models for each user interaction edge type (user-to-album, user-to-track, and user-to-artist), as well as for each model type (RHGNN and RGCN), the accuracy, precision, and error will be compared between the model types. Additionally, a comparison of these six model types can be compared for each type of LastFM1b subset, for each number of users able to be computed.

To put these evaluations into perspective, the evaluation could provide statistical results for every model type (2 possible types), every edge type (3 possible types), and every subset type (6 possible types), for a total of 36 unique results for a full evaluation on the LastFM1b data loader with all possible users.


\subsection{Evaluation of the RHGNN Heterogeneous Recommendation}

As the result of the contributions made in throughout this thesis, and the trained models that are compiled for link prediction of the different user interaction edge types. The final task is form recommendations for any user in the computed graph. To preform evaluation on these recommendations, different evaluation metrics must be utilized to form an understanding on the recommendation performance. 

Since the recommendations for any LastFMb1b input graph is computed by forming predictions from the precompiled link prediction models. Each link prediction model, in addition to the standard evaluation metrics as mentioned above, will require additional evaluation of the predictions.

To evaluate the link prediction model as a recommendation model, the prediction for all unseen edges that model predicts for will be ranked such that, for the top-K recommendations the different models can be evaluated for their accuracy, precision, diversity, coverage, and novelty @K recommendations.


The significance of evaluating a recommendation system in this manner is justified when mentioning that accuracy is not an expressive measurement of recommendation systems. Rather that, accuracy is an adequate measurement of relevancy, diversity is an adequate measurement of uniqueness, precision is a measurement of consistency, novelty is a measurement of newness, and coverage is a measurement of the capable reach of recommendation systems. These descriptive statistics will be able to provide explanation for the types of recommendation received using this approach.

Finally, as a special case, each of these model’s top-K recommendations can be combined and ranked into a multi target recommendation list for users which can be evaluated in a similar way. The unique difference of this approach is the introduction of additional metrics to evaluate the distributions of unique types of track, album, artists can be observed in the recommendation listings.






\newpage
