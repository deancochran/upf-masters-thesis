

\section{LastFM1b Link Prediction as recommendation}

Recommendation systems are not simply a link prediction model, however there are some principal correlations that graph based recommendation systems share with graph-based link prediction tasks. As mentioned prior there are constantly evolving methods for graph based recommendation systems \cite{Gao2021}. Specifically, HGNN recommendations models are in a constant state of change. To this extent, the objective of the thesis is to make artist, album, or track recommendations for users. Therefore, a scoring functioning to evaluate the probability of all unseen edges from a source node user, to an artist, album, or track destination node is required. There are different scoring techniques to identify a more expressive likelihood score of an edge existing.\cite{Kumar20} However, given that the implementation of three deep learning link predicting models have been proposed to score the likelihood of unseen edges for user-to-album, user-to-track, and user-to-artist interaction, these probabilities can be utilized as a likelihood score for personalized recommendation.


\subsection{Link Prediction as recommendation}

As mentioned above the link prediction models that are to be trained to predict specifically, the existence of unseen user-to-album, user-to-track, and user-to-artist interactions will be used to make recommendations for all user nodes in each graph. In order create recommendations, the highest preforming link prediction model for each link type (user-to-album, user-to-track, and user-to-artist), will iterate over all the specific user and destination node pairs relative to the link type. For every unique pairing of user and destination node, the respective trained link prediction model will score the likely hood the edge's likelihood of existing. Once complete, a selection of the top-K scored edges of a specified edge type for any user can be presented as recommendations. This process can be completed for ever edge type with a user source node (user-to-album, user-to-track, and user-to-artist).

Additionally, depending on the scoring likelihood of the top-K recommendations for each edge type, a ranked combination list of the three recommendation lists at top-K, could perform a multi-targeted recommendation for any specified user in the LastFM1b data set.
