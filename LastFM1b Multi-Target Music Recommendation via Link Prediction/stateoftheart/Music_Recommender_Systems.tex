\section{Music Recommendation Systems}
Implementing recommendation systems to simplify choice amidst billions of possible decisions that a particular user faces is no new problem. These technologies have existed for decades to filtering the overwhelming amount of information that exists, and forecast a users possible interactions with a particular item. 

Traditionally, recommendation systems have utilized Content-Based filtering and Collaborative-Based filtering methods to recommend items to users by utilizing past user interactions. However, these traditional recommendation systems have been discovered to be constrained by the academic community. These common recommendation constraints include, the cold start problem (when a system wants to recommend items to a user with no profile information) and data sparsity (when a system has many users and many items, but little information to compute relevant recommendations), and many others. \cite{Guo2020}

While these constraints are fundamental to understanding recommendation system limitations, the academic community has developed new machine learning techniques and models that are capable of alleviating these constraints while providing accurate and relevant recommendations. The key distinction between these machine learning methods and the traditional content or collaborative filtering methods lies within the ability of machine learning models to understand highly complex interactions in the graph and then precisely reflect the user's preferences within its recommendations. 

Particularly, with the increase of musical streaming platforms in the last decade, listening to the radio has been becoming a lost way to listen to music. This is mainly due to the fact that music streaming involves generating music recommendations for its streaming users, giving users a more immersive experience listening to music. In fact, providing a personalized streaming service for music streaming the most part expected for a private music streaming service. Amidst the various different industries that utilize recommendation systems however, the scope of music recommendation systems are different from other multimedia recommendation systems.

In comparison to other industry domains like movies or e-commerce, music has a short consumption duration, with billions ways for users to listen to media on the service. The fundamental time it takes to interact with a single track and abundance of possible tracks available implies that recommending a few songs that do not perfectly fit the user's taste will not affect the user experience in a detrimental way. Even repeated recommendations can be appreciated by the user. \cite{Riegler2019}. 

These unique characteristics of MRS standout from other industry specific recommendation systems. These particular characteristics in fact have even led to research in Sequential recommendation, which leverages the time a user listens to content as a primarily indication of what the use would like to be recommended. \cite{Fang2020}, Session based recommendation leverages the style of each session as a primarily indication of what the user would like to be recommended. \cite{Rijke_2019} 

Many recent developments in recommendation systems have shown that usage  of machine learning and artificial intelligence  models can not only alleviate the pitfalls of traditional content and collaborative filtering methods, but improve the performance of traditional methods.

