\chapter{Conclusions}

From the preliminary findings the are some key observations that can be made from each of the results sections. In sequential order, the data loader provides optimal flexibility and customize to provide transparency into the programs architecture. The final compiled graph from the data loader removes a significant amount of artists, albums, and tracks. However, the information loss is reduced to a minimum with much of the information retained in the resulting fully connected graph.

The data loader has been tested for optimal performance, and adjusted to utilize fast computation methods. As mentioned, all the the data cleaning and preparation has been compiled from this data set to fit in a ~5GB binary output file. Additionally, once compiled the graph can run multiple post processing operations that do not include filtering through the raw input files. As seen the data loader has a significant amount of adjustments to account for. The promised information that can be acquired from the implementation, while can be acquired, is not efficient enough to be ran on the research device that the data loader was built with. Therefore, as mentioned, many of the subsets that were capable of being compiled could be fit into memory to be validated through model testing.


The link models and recommendation algorithm followed this limited result with similar barriers. Without the necessary information to compute link prediction models, it was not possible to preform the necessary recommendation. As mentioned however, there was one subset and one model to be computed. Given the full LastFM1b graph, with users artists, and genres. The user to artist link prediction model that was given the proper compiled information was able to preform a very high level in both the implemented model (RHGNN) and outsourced model (RGCN). To also note the comparison of the models performance, it provides additional justification to the results found in the RHGNN paper. This is because in the RHGNN paper there was a high performance measured for the proposed model, than that of the RGCN model.

However, to conclude upon the recommendations of this model is challenging for two distinct reasons. The first being that the models' performance whilst effective and notably optimistic, there are some significant factors that need to be validated before such a high performing result is excepted. Secondly, the recommendation algorithm whilst working does not have evaluation metrics like diversity, novelty, coverage, or accuracy to evaluate the true performance of the single models recommendation performance.

Lastly, it is difficult to believe that there will be capabilities to perform multi-target recommendation with compiled forms of subsets with more contextual and semantic information that just users, artists, and genres.

\section{Discussion}
To over a brief discussion on the work presented in this paper. I find the task at hand of identifying the significance of incorporating tracks, and albums, into listening event data sets to be particularly significant. In many deep graph learning papers, they do not utilize this additional information, and they do not mention its existence when using the LastFM data set. I find that preforming graph based recommendation on a bipartite graph of users and artists to be resourceful, and when researchers began to incorporate tags and genres into the data set. Note worthy performance increases were shown. This is particularly why I have personal interest in incorporating more data into.

To briefly mention some pitfalls of this approach, I find that due to the unforeseen storage allocation errors, a smaller data set like the 30Music data set could have been used to preform heterogeneous graph neural network tasks. This data set offers more features than the LastFM1b data set and could possible provide a smaller footprint, given there are some hidden edge connections that could exacerbate the storage requirements. In short, it's difficult to know if this methodology would be a good approach for other data sets since the distribution of edge connections between entities is often not listed in these large databases. Additionally, the LastFM2b was released within the last year, and while I could adjusted this project to account for this newer, larger data set, I believe this was the right decision for someone very unacquainted with the LastFM data sets.

\section{Future Work}
As for the decisions to be made from here on. I believe that this project has the capability of providing the promised results, given the opportunity to utilize a device with powerful enough hardware to compute the necessary node embed dings for large graphs. Each as it has not been tested on larger subsets, I am not for certain of the required storage space to process the graph, however I do know that it is not larger than the maximum storage of the graph object which by design of the data set is less than 8GB.

However, given the likely occurrence of utilizing larger, more powerful devices as unlikely, validating the user to model of the small subset of the LastFM1b is a primary step in justifying why this data set should be used. In doing so the evaluation metrics for recommending artists to users will also be completed.

Aside from training, the data loading repository and preliminary masters thesis code has already been published on GitHub. 


https://github.com/deancochran/upf-masters-thesis

https://github.com/deancochran/DGL\textunderscore{}LFM1b



Therefore providing significant versioning of the repositories used in this thesis will justify the commitment to reproducible code many published graph neural papers do not submit.


Lastly, as mentioned in the discussion the LastFM2b data set has been developed. Though, the data loader for LastFM1b has been tested, it has not been ran completely for a full graph evaluation. It begs the question of whether it could be a better use of my knowledge of custom data loaders for the deep graph library framework, to build the data loader for the LastFM2b data set as well

\newpage