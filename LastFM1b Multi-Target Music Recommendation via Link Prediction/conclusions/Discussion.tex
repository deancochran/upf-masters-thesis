\section{Discussion}
To over a brief discussion on the work presented in this paper. I find the task at hand of identifying the significance of incorporating tracks, and albums, into listening event data sets to be particularly significant. In many deep graph learning papers, they do not utilize this additional information, and they do not mention its existence when using the LastFM data set. I find that preforming graph based recommendation on a bipartite graph of users and artists to be resourceful, and when researchers began to incorporate tags and genres into the data set. Note worthy performance increases were shown. This is particularly why I have personal interest in incorporating more data into.

To briefly mention some pitfalls of this approach, I find that due to the unforeseen storage allocation errors, a smaller data set like the 30Music data set could have been used to preform heterogeneous graph neural network tasks. This data set offers more features than the LastFM1b data set and could possible provide a smaller footprint, given there are some hidden edge connections that could exacerbate the storage requirements. In short, it's difficult to know if this methodology would be a good approach for other data sets since the distribution of edge connections between entities is often not listed in these large databases. Additionally, the LastFM2b was released within the last year, and while I could adjusted this project to account for this newer, larger data set, I believe this was the right decision for someone very unacquainted with the LastFM data sets.