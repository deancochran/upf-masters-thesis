
\chapter{Results}   
As a continuation from the methodologies and approach of the thesis. The results chapter will offer further technical insight into the findings from the methods chosen. Specifically, when computing the data loader for the LastFM1b, the results of the approach should present the findings from compiling such a large  data collection. Outlining the amount of information loss, the total number of edges, nodes, and connections in the graph. As well as information concerning storage space, and time it takes to compile a full heterogeneous graph into memory.

Secondly the findings from the link prediction models should be presented. The parameters chosen for the best preforming models should be outlined, and the methods for tuning the models should be noted. Since this also a comparative study between the RHGNN and RGCN models, so listing the performance of the models for the data collection should be outlined, as well as interpreted.

Additionally, the different subsets capable of being computed from the LastFM1b data set should be utilized for additional comparisons for further validation of the LastFM1b data set as this thesis aims to justify the need for additional contextual and semantic information utilized for music recommendation 

Finally, the findings from the recommendations computations should be discussed. As this is not a recommendation performance analysis, accuracy of the recommendations will not be the only covered evaluation metric. Rather all the evaluation measures as mentioned in the methods chapter should be applied to outline interpretative results and offer insights on the link prediction models.

\section{LastFM1b Data Loader}

\subsection{Preliminary Findings}

From the data loader, there have been a significant amount of findings and logistic barriers that are most definitely worth mentioning a preliminary results section.

From the multiple modes of the LastFM1b data loader states, when computing the full LastFM1b heterogeneous graph. Some basic count measurements can provide a few additional measurements not directly mentioned in the distributional analysis of the LastFM1b. In regard to run and storage time, the time it takes to fully compile a graph is about ~3hours, as there are a significant amount of bad instances that exist within the LastFM1b. 

\begin{table}[!ht]
\renewcommand{\arraystretch}{1.50}
\caption{The number of nodes the full LFM1b data loader can capture}
\label{tablePCA}
\centering
\begin{tabular}{| c | c |}
\hline
\bfseries Node Type & \bfseries Number of Nodes \\
\hline\hline
User & (~120K)  \\
\hline
Artsit & (~3M)\\
\hline
Album & (~15M)  \\
\hline
Track & (~32M)  \\
\hline
Genre & (21)  \\
\hline
\end{tabular}
\end{table}

\begin{table}[!ht]
\renewcommand{\arraystretch}{1.50}
\caption{The number of edges the full LFM1b data loader can capture}
\label{tablePCA}
\centering
\begin{tabular}{| c | c |}
\hline
\bfseries Edge Type (no reverse edges) & \bfseries Number of Edges \\
\hline\hline
User -> Artist & (61411336)  \\
\hline
User -> Album & (?)\\
\hline
User -> Track & (?)  \\
\hline
Artist -> Genre & (414379)  \\
\hline
Album -> Artist & (14184326)  \\
\hline
Track -> Artist & (27258365)  \\
\hline
\end{tabular}
\end{table}

From these findings you can observe the first barrier as not being able to find the number of edges that exist between user and track, user, and album. While this is the correct assumption, I rather just have not had time to compute these individual statistics due to large amount of computational time it takes to compile the full graph, even without the metapath2vec module.

Additionally, it should be noted that when using metapath2vec, there are tremendous allocation errors due to the necessary storage required to store a graph representation in local CPU memory. To account for this, the subset module was utilized to not only split the graph by node types if desired but specify the number of users to collect for a compiled interaction graph. Since users have the most edges and inter connections within the LastFM1b data set, this adjustment significantly reduces the allocation space required to compile a full graph.

However, in many, if not most cases, the number of users has to be below 50 to compute a full compiled graph with metapath2vec node embedding representations. Though this is a significant reduction of the information provided in the usable number of users for link prediction and recommendation. There is no sacrifice on the number of artists, albums, or tracks available in the graph, for any subset. This, paired with the subset model, which when N users are specified, finds users to maintain the original demographic distribution of the original LastFM1b data set prevents as much information loss by making a subset as possible.

Yet another major barrier for this data loader, finds another allocation error when computing graphs that utilize more than one node type, ergo a LastFM1b data set of artists and tracks, artists, and albums, or just albums, or just tracks for any specified number of users. This has been the most significant
barrier for provided quality, non-preliminary results on the deep graph library data loader.

Since the data loader works, but the allocation errors prove to be cumbersome, this evaluation of the thesis may possibly be inconclusive if additional storage space is not reallocated permitted to be utilized for this research project.

As a result, from these limitations, the data loader though not able to compile the full graph, has found ample space to create a compiled version of itself, if and only if the following is used: a small number of users, artists, and genres. Though this graph prevents further analysis on the unique impact of adding tracks or albums music recommendation graphs, it is still able to compute information on one of the data loading subsets that can be utilized for one type of link prediction model that we be attempted later.



\section{LastFM1b Link Prediction}


\subsection{Preliminary Findings}

From each type of subset created (just one) only one link prediction model for RGCN or RHGNN can be made, user to artist link prediction, on a LastFM1b data set.

This significantly limits the results from this section of the chapter as the results for the link prediction models can only be created from the subset of users, artists, and genres, where the number of users is below 50.


\begin{table}[!ht]
\renewcommand{\arraystretch}{1.50}
\caption{Evaluation of User to Artist LFM1b Link Prediction on Artist Genre Subset}
\label{tablePCA}
\centering
\begin{tabular}{| c | c | c | c | c | c |}
\hline
\bfseries Edge Type & \bfseries Accuracy & \bfseries Precision & \bfseries MAE & \bfseries RMSE \\
\hline
\hline
RGCN & 0.9202 & na & na & na   \\
\hline
RHGNN & 0.9873 & 0.9905 & 0.0925 & 0.291   \\
\hline
\end{tabular}
\end{table}

From these overly optimistic results, the RGCN model, implemented through the OpenHGNN GitHub repository, and the implemented RHGNN model share an extremely accurate and precise prediction capabilities for the LastFM1b user to artist edge. Though To accompany this observation, it begs, let alone warrants the use of a visualization of the loss curve.


\begin{figure}[!ht]
    \includegraphics[clip,width=\columnwidth]{Figures/listened_to_artist_ap_plot.png}% 
\caption{This is the RHGNN precision performance over time}
\label{fig:timeseries}
\end{figure}

\begin{figure}[!ht]
    \includegraphics[clip,width=\columnwidth]{Figures/listened_to_artist_auc_plot.png}% 
\caption{This is the RHGNN accuracy performance over time}
\label{fig:timeseries}
\end{figure}

\begin{figure}[!ht]
    \includegraphics[clip,width=\columnwidth]{Figures/listened_to_artist_mae_plot.png}% 
\caption{This is the RHGNN Mean Absolute Error performance over time}
\label{fig:timeseries}
\end{figure}

\begin{figure}[!ht]
    \includegraphics[clip,width=\columnwidth]{Figures/listened_to_artist_rmse_plot.png}% 
\caption{This is the RHGNN Root Mean Squared Error performance over time}
\label{fig:timeseries}
\end{figure}

\begin{figure}[!ht]
    \includegraphics[clip,width=\columnwidth,]{Figures/listened_to_artist_loss_plot.png}% 
\caption{This is the RHGNN Loss performance over time}
\label{fig:timeseries}
\end{figure}

From these figures is can be observed that the model appears to be generalizing the training information well and not over-fitting. However, do to the noticeably high prediction scores of the accuracy in precision. The overly optimistic results do appear to show ideal training results. While proper results are not necessarily a requirement, it is a large concern that should be validated with further testing. Current ideas of why the accuracy and precision is so high is due to the small subset of users that are required to compile the graph into memory. Additionally, as it could be possible, there may be small data leakage inside the training step that could have allowed the model to learn the downstream validation or testing prediction tasks during training. Further studies, or the support of more memory for allocation would greatly support the completion of these findings.
\section{LFM1b Recommendation}


\subsection{Preliminary Findings}

To arrive at an inconclusive results section briefly, and unfortunately. The recommendations for graphs that contain more than just users, artists or genres as mentioned above, have not be able to be compiled. This prevents link prediction models from being computed for the user to album or user to track links that exists within the LastFM1b dataset.

As a directly effect of not being able to have a heterogeneous graph with information of tracks or albums, there are no models to predict these edges. Therefore, at the time being, there is only the capability to compute a link prediction model of user to artist edges with a subset of less that 50 users.

The indirect result of this effect prevents any recommendations to be made for users involving tracks or albums. Lastly this prevents the end goal of the thesis to produce a multi target music recommendation list for users utilizing the trained link prediction models.

As a final remark, the algorithm allows can compute recommendations for any given user, and they can be mapped from their unique number id back to the artist’s name, when provided the link prediction model. The link prediction model of just user to artist, while not having recommendation evaluation metrics recorded at this, is able to provide a user listening history, and the top@K recommendations for the given sample edge type and corresponding model.


\newpage


